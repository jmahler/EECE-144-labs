% # COPYRIGHT:
%
% Copyright (C) 2011 Jeremiah Mahler <jmmahler@gmail.com>.
% Permission is granted to copy, distribute and/or modify this document
% under the terms of the GNU Free Documentation License, Version 1.3
% or any later version published by the Free Software Foundation;
% with no Invariant Sections, no Front-Cover Texts, and no Back-Cover Texts.
% A copy of the license is included in the file "fdl-1.3.txt".
%
\documentclass[12pt]{article}
%\usepackage{mslapa}
\usepackage{hyperref}
\usepackage{amsmath}
\usepackage{graphicx}
\usepackage{ulem}
\usepackage{vmargin}
\usepackage{tabularx}
\usepackage{sectsty}
\usepackage{pbox}
\usepackage{bigstrut}
\usepackage{enumerate}
\usepackage{parskip} % add spaces between paragraphs
\input kvmacros % Karnaugh Maps and Veitch charts
%\usepackage{cleveref}
\usepackage{verbatim}

\setpapersize{USletter}
\setmarginsrb{1.0in}{1.0in}{1.0in}{1.0in}{0in}{0.25in}{0in}{0.20in}

\sectionfont{\normalsize}
\subsectionfont{\normalsize}

% configure \bigstrut size
% This configures spacing above and below rows in a tabularx.
%\renewcommand{\bigstrutjot}{6pt}

\renewcommand{\bigstrutjot}{2.0\jot}

\setlength{\parindent}{0in}

\raggedright

\begin{document}

% {{{ Cover Page
\centerline{\bf EECE 144}
\centerline{\bf Fall 2011}
\centerline{\bf}
\centerline{\bf Lab Report \#11}
\centerline{\bf Section 4}
\centerline{\bf 11/16/2011}
% signature area
\begin{center}
\begin{tabularx}{\textwidth}[b]{X l l}
Submitted by: Jeremiah Mahler & & \\
Signature & Printed Name & Date \\
\hline
\multicolumn{1}{|X|}{} & \multicolumn{1}{|l|}{\bigstrut \bf Jeremiah Mahler} & \multicolumn{1}{|l|}{\bf Nov 16, 2011} \\
\hline
\multicolumn{1}{|X|}{} & \multicolumn{1}{|l|}{\bigstrut \bf Marvanee Johnson} & \multicolumn{1}{|l|}{\bf Nov 16, 2011} \\
\hline
\end{tabularx}
\end{center}
% }}}

% {{{ Description/Objectives
\section{Description/Objectives}

The objective of this lab is to design a three bit binary counter
that iterates over the following non-sequential sequence
in ascending from top to bottom when $X = 0$ and in
descending order when $X = 1$.
The design must use one JK, one T, and one D flip-flop.

\begin{center}
\begin{tabular}[t]{cccc}
0 & 0 & 0 \\
0 & 1 & 0 \\
1 & 1 & 0 \\
0 & 1 & 1 \\
1 & 0 & 1 \\
0 & 0 & 1 \\
1 & 0 & 0 \\
1 & 1 & 1 \\
\end{tabular}
\end{center}

% }}}

% {{{ Procedure
\section{Procedure}
\label{sec:procedure}

To implement this counter three flip flops will be used.
Any type of flip flop can be used in any order but in this
specific implementation we will use a JK flip-flop for the
most significant bit followed by a T and a D.

The first step is to build a state table which describes all
the required counter states along with the flip-flop inputs
necessary to produce the desired outputs as shown in Table \ref{tbl:st}. 
From this table the mapping from the inputs
($Q_2$, $Q_1$, $Q_0$) to gate inputs ($J_2$, $K_2$, $T_1$, etc) can be determined.

\begin{table}
\center
\begin{tabular}[t]{r|cccc|ccc|cc|c|c}
n & $X$ & $Q_2$ & $Q_1$ & $Q_0$ & $Q_2^+$ & $Q_1^+$ & $Q_0^+$ & $J_2$ & $K_2$ & $T_1$ & $D_0$ \\ 
\hline
0 & 0 & 0 & 0 & 0   & 0 & 1 & 0   & 0 & x   & 1   & 0 \\ 
1 & 0 & 0 & 0 & 1   & 1 & 0 & 0   & 1 & x   & 0   & 0 \\ 
2 & 0 & 0 & 1 & 0   & 1 & 1 & 0   & 1 & x   & 0   & 0 \\ 
3 & 0 & 0 & 1 & 1   & 1 & 0 & 1   & 1 & x   & 1   & 1 \\ 
4 & 0 & 1 & 0 & 0   & 1 & 1 & 1   & x & 0   & 1   & 1 \\ 
5 & 0 & 1 & 0 & 1   & 0 & 0 & 1   & x & 1   & 0   & 1 \\ 
6 & 0 & 1 & 1 & 0   & 0 & 1 & 1   & x & 1   & 0   & 1 \\ 
7 & 0 & 1 & 1 & 1   & 0 & 0 & 0   & x & 1   & 1   & 0 \\ 
8 & 1 & 0 & 0 & 0   & 1 & 1 & 1   & 1 & x   & 1   & 1 \\ 
9 & 1 & 0 & 0 & 1   & 1 & 0 & 1   & 1 & x   & 0   & 1 \\ 
10 & 1 & 0 & 1 & 0   & 0 & 0 & 0   & 0 & x   & 1   & 0 \\ 
11 & 1 & 0 & 1 & 1   & 1 & 1 & 0   & 1 & x   & 0   & 0 \\ 
12 & 1 & 1 & 0 & 0   & 0 & 0 & 1   & x & 1   & 0   & 1 \\ 
13 & 1 & 1 & 0 & 1   & 0 & 1 & 1   & x & 1   & 1   & 1 \\ 
14 & 1 & 1 & 1 & 0   & 0 & 1 & 0   & x & 1   & 0   & 0 \\ 
15 & 1 & 1 & 1 & 1   & 1 & 0 & 0   & x & 0   & 1   & 0 \\ 
\end{tabular}
\caption{State table for  the 3-bit counter using a JK,
T and D flip-flop in order with the JK as the most significant bit.
The "don't care" values are denoted by an x.}
\label{tbl:st}
\end{table}

% {{{ bit $Q_2$ using JK flip-flop
\subsection{bit $Q_2$ using JK flip-flop}

The Karnaugh Maps for bit $Q_2$ are shown in Figure \ref{fig:kmapQ2}.
From this table we can see that $J_2$ and $K_2$ are disjoint so we
can build a single function for them both.
Grouping ones to form implicants results in the SOP equation
\footnote{The slash(/) used here is meant to indicate "either or"
and should not be confused with division.}
\begin{align*}
J_2/K_2 &= X' Q_1 + X' Q_0 + Q_2 Q_1 Q_0' + Q_2' Q_0 + X Q_1'
\end{align*}

\begin{figure}
\begin{tabular}{cc}
\karnaughmap{4}{$J_2:$}{{$X$}{$Q_2$}{$Q_1$}{$Q_0$}}{0111XXXX1101XXXX}{}
&
\karnaughmap{4}{$K_2:$}{{$X$}{$Q_2$}{$Q_1$}{$Q_0$}}{XXXX0111XXXX1110}{}
\\
\multicolumn{2}{c}{
\karnaughmap{4}{$J_2/K_2:$}{{$X$}{$Q_2$}{$Q_1$}{$Q_0$}}{0111011111011110}{}
}
\end{tabular}
\caption{Karnaugh Maps for bit $Q_2$ used to describe the mapping from
the inputs ($Q_2$, $Q_1$, $Q_2$, $X$) to the input of the
JK flip-flop ($J_2$, $K_2$).
In this case $J_2$ and $K_2$ are disjoint so they can be merged in
to a single table ($J_2/K_2$).
}
\label{fig:kmapQ2}
\end{figure}
% }}}

% {{{ bit $Q_1$ using T flip-flop
\subsection{bit $Q_1$ using T flip-flop}

The Karnaugh Map for bit $Q_1$ is shown in Figure \ref{fig:kmapQ1}.
Grouping ones to form implicants results in the SOP equation
\begin{align*}
T_1 &= X' Q_1' Q_0' + X' Q_1 Q_0 + X Q_2 Q_0 + X Q_2' Q_0'
\end{align*}

\begin{figure}
\center
\karnaughmap{4}{$T_1:$}{{$X$}{$Q_2$}{$Q_1$}{$Q_0$}}{1001100110100101}{}
\caption{Karnaugh Map for bit $Q_1$ used to describe the mapping from
the inputs ($Q_2$, $Q_1$, $Q_2$, $X$) to the input of the T flip-flop $T_1$.}
\label{fig:kmapQ1}
\end{figure}
% }}}

% {{{ bit $Q_0$ using D flip-flop
\subsection{bit $Q_0$ using D flip-flop}

The Karnaugh Map for bit $Q_0$ is shown in Figure \ref{fig:kmapQ0}.
Grouping ones to form implicants results in the SOP equation
\begin{align*}
D_0 &= Q_2 Q_1' + X' Q_2 Q_0' + X Q_1' + X' Q_2' Q_1 Q_0
\end{align*}

\begin{figure}
\center
\karnaughmap{4}{$D_0:$}{{$X$}{$Q_2$}{$Q_1$}{$Q_0$}}{0001111011001100}{}
\caption{Karnaugh Map for bit $Q_0$ used to describe the mapping from
the inputs ($Q_2$, $Q_1$, $Q_2$, $X$) to the input of the D flip-flop $D_0$.}
\label{fig:kmapQ0}
\end{figure}
% }}}

% {{{ circuit construction
\subsection{circuit construction}

After all the equations for each bit has been found these
be combined together to produce a complete solution that works
for all bits and implements the 3-bit, non-sequential counter.
In this case Logisim\cite{LOGISIM} will be used to build and simulate the circuit.

This implementation has many interconnections so its construction
can be quite tedious.
The resulting circuit is shown in Figure \ref{fig:circ}.
Every single combination should be verified against the
state table (Table \ref{tbl:st}) to make every connection is right.

In the event of an error it is best to methodically find the error rather
than removing all or some and starting over.
Starting over does nothing to find the problem and it is possible that
the problem could be built again if there is an error in the design.
A good strategy for finding problems is to try and isolate the problem.
For example, if the bit $Q_2$ is wrong it would make sense to check
the JK flip-flop since that is the bit it controls.
It most likely would be a waste of time to examine the other flip-flops.

\begin{figure}
\center
\includegraphics[scale=0.5]{lab11-circuit}
\caption{Circuit diagram of 3 bit-counter.}
\label{fig:circ}
\end{figure}

% }}}

% }}}

% {{{ Observations
\section{Observations}

\samepage
It is possible to build a 3-bit non-sequential counter using a JK,
T, and D flip-flops.
The counter, when simulated in Logisim, reproduced the desired sequence
in both the up and down directions.

Due to the tedious nature of this design several errors did
occur while building the circuit which had to be solved.
The strategy of trying to isolate the problem to a specific bit
or part of a function worked well.

% }}}

% {{{ Conclusion
\section{Conclusion}

This lab was a success in designing a three bit non-sequential counter.
When simulated in Logisim it reproduced all the desired values without
any problems.

% }}}

%\clearpage

\renewcommand*{\refname}{\vspace{-8mm}}
\section{References}
%%\bibliographystyle{plain}
%%\bibliographystyle{mslapa}
\bibliographystyle{ieeetr}
\bibliography{../references}

\end{document}

% vim:foldmethod=marker
