
%
% # INTRODUCTION:
%
% This is a template (skeleton) file in LaTeX used for lab write-ups.
%
% Specifically this was used for the class Logic Design Fundamentals (EECE 144)
% taught by Kurtis Kredo [http://www.ecst.csuchico.edu/~kkredo/]
% during the Fall 2011 semester at CSU Chico [www.csuchico.edu].
% 
% ## LaTeX
%
% This file is written for LaTeX [http://www.latex-project.org/]
% which is used to process this file in to a completely formatted
% document.
%
% If you are unfamiliar with LaTeX it can seem daunting at first
% (as with anything new) but there are many benefits.
% Imagine writing a document in Word except without
% having to worry about the tedious things such as line breaks,
% indentation, table of contents, appendices, font styles,
% heading sizes, citations/references and page numbers.
% LaTeX lets you focus on the content without
% worrying about the tedious details.  It is also great at
% producing publication quality mathematical formulas.
% 
% If you are collaborating with someone else you can simply edit
% the sections and paragraphs in this file as needed.
%
% To process this file use a command such as 'rubber'.
%
%   bash$ rubber skel.tex
%   (output to skel.dvi)
%   bash% rubber --pdf skel.tex
%   (output to skel.pdf)
%
% # AUTHORS (of this template):
%
%   Jeremiah Mahler <jmmahler@gmail.com>
%   https://www.google.com/profiles/jmmahler#about 
%
% # COPYRIGHT:
%
%   Copyright (C)  2011 Jeremiah Mahler <jmmahler@gmail.com>.
%   Permission is granted to copy, distribute and/or modify this document
%   under the terms of the GNU Free Documentation License, Version 1.3
%   or any later version published by the Free Software Foundation;
%   with no Invariant Sections, no Front-Cover Texts, and no Back-Cover Texts.
%   A copy of the license is included in the file "fdl-1.3.txt".
%

\documentclass[12pt]{article}
%\documentclass[10pt]{article}

%\usepackage{mslapa}
\usepackage{hyperref}
\usepackage{amsmath}
\usepackage{graphicx}
\usepackage{ulem}
\usepackage{vmargin}
\usepackage{tabularx}
\usepackage{sectsty}
\usepackage{pbox}
\usepackage{bigstrut}
\usepackage{enumerate}

\setpapersize{USletter}
\sectionfont{\normalsize}

\begin{document}

% {{{ Cover Page

\centerline{\bf EECE 144}
\centerline{\bf Fall 2011}
\centerline{\bf}
\centerline{\bf Lab Report \#0}
\centerline{\bf Section 8}
\centerline{\bf 9/7/2011}

% signature area
% TODO - how to get more vertical space from \bigstrut?
\begin{center}
\begin{tabularx}{\textwidth}[b]{X X l}
Submitted by: & & \\
Signature & Printed Name & Date \\
\hline
\multicolumn{1}{|X|}{} & \multicolumn{1}{|l|}{\bigstrut \bf A Student} & \multicolumn{1}{|l|}{\bf Sep 7, 2011} \\
\hline
\multicolumn{1}{|X|}{} & \multicolumn{1}{|l|}{\bigstrut \bf B Student} & \multicolumn{1}{|l|}{\bf Sep 7, 2011} \\
\hline
\end{tabularx}
\end{center}
% }}}

\section{Description/Objectives}

This “lab assignment” illustrates how to write a lab report, using a quiz study
session as an example. The objective is to learn to write good lab reports and
gain some tips on how to study for EECE 144.

\section{Procedure}

We began by collecting our notes and questions together. The collected
questions were ordered based on interest and student uncertainty.
For each question, we:

\begin{enumerate}[a.]
\item Reviewed our class notes and the appropriate textbook sections until
everyone had a basic understanding of the material.
\item Quizzed each other by solving the textbook problems that had answers
provided in the back of the book.
\item In cases where a member still had uncertainty, we created additional
problems for group discussion and as solution examples.
\end{enumerate}

During the study session, one student tried to solve an assigned homework
problem with the group. The other group members reminded everyone that
homework assignments must be done individually and to solve them as a group
would be academic misconduct and lead to unpleasant results.

After resolving all outstanding questions, the group performed a final quiz by
asking made-up questions on random topics. Once all members felt comfortable
with the topics, we went home to get plenty of sleep before the quiz.

\section{Observations}

We found that meeting as a group to study helped each member learn the course
material. Each student brought an understanding of certain topics and a
weakness in other topics. As a group, we had a combined knowledge that no
individual student possessed.
\nocite{LOGISIM}

During the study session, we also learned that creating questions and explaining
material to other students was an excellent test of understanding. Often we
thought we knew a subject until we had to explain it to another student.

%\section{Analysis}
%
%Analyze the observations.

\section{Conclusion}

Studying in groups supported and improved the understanding of all students
involved. We feel more confident and better prepared after the study session
than we did earlier in the semester.

The example lab report also provided welcome guidance on how to gain the most
from lab and better prepare our lab reports.

% flush all the figures
%\clearpage

% References (if needed)
% If you don't have any references
% comment out this section.
%\pagebreak
% for References in table of contents
\renewcommand*{\refname}{}
\section{References}
% choose one of the styles
%\bibliographystyle{plain}
%\bibliographystyle{mslapa}
\bibliographystyle{ieeetr}
\bibliography{references}  % references.bib

% Appendix (if needed)

\end{document}

% vim:foldmethod=marker
