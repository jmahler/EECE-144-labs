
%
% # INTRODUCTION:
%
% This is a template (skeleton) file in LaTeX used for lab write-ups.
%
% Specifically this was used for the class Logic Design Fundamentals (EECE 144)
% taught by Kurtis Kredo [http://www.ecst.csuchico.edu/~kkredo/]
% during the Fall 2011 semester at CSU Chico [www.csuchico.edu].
% 
% ## LaTeX
%
% This file is written for LaTeX [http://www.latex-project.org/]
% which is used to process this file in to a completely formatted
% document.
%
% If you are unfamiliar with LaTeX it can seem daunting at first
% (as with anything new) but there are many benefits.
% Imagine writing a document in Word except without
% having to worry about the tedious things such as line breaks,
% indentation, table of contents, appendices, font styles,
% heading sizes, citations/references, page numbers.
% LaTeX lets you focus on the content without
% worrying about the tedious details.  It is also excellent for
% producing mathematical formulas.
% 
% If you are collaborating with someone else you can simply edit
% the sections and paragraphs in this file as needed.
%
% To process this file use a command such as 'rubber'.
%
%   bash$ rubber skel.tex
%   (output to skel.dvi)
%   bash% rubber --pdf skel.tex
%   (output to skel.pdf)
%
% # AUTHORS (of this template):
%
%   Jeremiah Mahler <jmmahler@gmail.com>
%   https://www.google.com/profiles/jmmahler#about 
%
% # COPYRIGHT:
%
%   Copyright (C)  2011 Jeremiah Mahler <jmmahler@gmail.com>.
%   Permission is granted to copy, distribute and/or modify this document
%   under the terms of the GNU Free Documentation License, Version 1.3
%   or any later version published by the Free Software Foundation;
%   with no Invariant Sections, no Front-Cover Texts, and no Back-Cover Texts.
%   A copy of the license is included in the file "fdl-1.3.txt".
%

\documentclass[12pt]{article}
%\documentclass[10pt]{article}

%\usepackage{mslapa}
\usepackage{hyperref}
%\usepackage{fancyhdr}
%\usepackage{lastpage}  % Page n of m, \pageref{LastPage}
\usepackage{amsmath}
\usepackage{graphicx}
\usepackage{ulem}       % \uline
\usepackage{vmargin}    % \setpapersize

\setpapersize{USletter}

\begin{document}

% {{{ Cover Page

\null
\thispagestyle{empty}
\vfill
\centerline{\LARGE Lab 1: Lab Introduction}
\vspace{0.3in}
\centerline{Jeremiah M. Mahler and Second Partner}
\centerline{Logic Design Fundamentals (EECE 144), CSU Chico}
%\centerline{\today}
\centerline{September 8, 2011}
\centerline{Kurtis Kredo, Instructor}
\vspace{5in}

% signature area
\begin{center}
\begin{tabular}{l l}
\uline{\hspace{2.5in}} & \uline{\hspace{2.5in}} \\
Signature (Jeremiah Mahler) & Signature (Second Partner) \\
& \\
& \\
\uline{\hspace{2.5in}} & \uline{\hspace{2.5in}} \\
Date & Date \\
\end{tabular}
\end{center}

\vfill  % fill up the rest of the page with spaces, shifts content up
\pagebreak
% }}}

% {{{ Table of Contents
\tableofcontents

\pagebreak
% }}}

\section{Objectives}

What is the goal of this lab?

\section{Procedure}

What was procedure you used in this lab?
Describe it in detail so that someone else you repeat
this procedure to confirm your results.

\section{Observations}

Describe any notable behaviour that was observed.

\section{Analysis}

Analyze the observations.

\section{Conclusion}

Describe your conclusions from this lab.
Was it successful?

% flush all the figures
\clearpage

% References (if needed)
% If you don't have any references
% comment out this section.
%\pagebreak
%\bibliographystyle{plain}
%\bibliographystyle{mslapa}
%\bibliography{main}  % main.bib

% Appendix (if needed)

\end{document}

% vim:foldmethod=marker
