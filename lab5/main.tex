
% # COPYRIGHT:
%
%   Copyright (C)  2011 Jeremiah Mahler <jmmahler@gmail.com>.
%   Permission is granted to copy, distribute and/or modify this document
%   under the terms of the GNU Free Documentation License, Version 1.3
%   or any later version published by the Free Software Foundation;
%   with no Invariant Sections, no Front-Cover Texts, and no Back-Cover Texts.
%   A copy of the license is included in the file "fdl-1.3.txt".
%

\documentclass[12pt]{article}
%\documentclass[10pt]{article}

%\usepackage{mslapa}
\usepackage{hyperref}
\usepackage{amsmath}
\usepackage{graphicx}
\usepackage{ulem}
%\usepackage{vmargin}
\usepackage{tabularx}
\usepackage{sectsty}
\usepackage{pbox}
\usepackage{bigstrut}
\usepackage{enumerate}
\usepackage{parskip}  % add spaces between paragraphs
\input kvmacros  % Karnaugh Maps and Veitch charts
\usepackage[table]{xcolor}

%\usepackage{cleveref}

%\setpapersize{USletter}
\sectionfont{\normalsize}
\subsectionfont{\normalsize}

% configure \bigstrut size
% This configures spacing above and below rows in a tabularx.
%\renewcommand{\bigstrutjot}{6pt}
\renewcommand{\bigstrutjot}{2.0\jot}

\setlength{\parindent}{0in}

\raggedright

\begin{document}

% {{{ Cover Page

\centerline{\bf EECE 144}
\centerline{\bf Fall 2011}
\centerline{\bf}
\centerline{\bf Lab Report \#5}
\centerline{\bf Section 4}
\centerline{\bf 10/5/2011}

% signature area
\begin{center}
\begin{tabularx}{\textwidth}[b]{X l l}
Submitted by: & & \\
Signature & Printed Name & Date \\
\hline
\multicolumn{1}{|X|}{} & \multicolumn{1}{|l|}{\bigstrut \bf Jeremiah Mahler} & \multicolumn{1}{|l|}{\bf Oct 5, 2011} \\
\hline
\multicolumn{1}{|X|}{} & \multicolumn{1}{|l|}{\bigstrut \bf Marvanee Johnson} & \multicolumn{1}{|l|}{\bf Oct 5, 2011} \\
\hline
\end{tabularx}
\end{center}
% }}}

\section{Description/Objectives}

The objective of this lab is to derive a simplified form
of a canonical SOP equation using a Karnaugh Map and
to implement it in hardware.

\section{Procedure}
\label{sec:procedure}

The canonical SOP form of the equation used in this lab is
given in Equation \ref{eq:cansop}.

\begin{align}
f(a, b, c) &= \sum m(0, 2, 3, 4, 6) \label{eq:cansop} \\
	    &= m_0 + m_2 + m_3 + m_4 + m_6 \notag \\
		&= a' b' c' + a' b c' + a' b c + a b' c' + a b c' \notag
\end{align}

The truth table produced from Equation \ref{eq:cansop}
is shown in Figure \ref{fig:tt}.
And the Karnaugh Map produced from the truth table
is shown in Figure \ref{fig:karnmap}.
Using the groupings produced by the Karnaugh Map results in
the simplifed Equation \ref{eq:simp}.
Finally the circuit definition is show in Figure \ref{fig:circuit}.

\begin{align}
f(a, b, c) &= c' + a'b \label{eq:simp}
\end{align}

\begin{figure}[!hbt]
\begin{center}

\begin{tabular}{lr}
  \begin{tabular}[t]{r|ccc|c}
Index&$a$&$b$&$c$&$f$\\
\hline
0 &0&0&0 &1\\
1 &0&0&1 &0\\
2 &0&1&0 &1\\
3 &0&1&1 &1\\
4 &1&0&0 &1\\
5 &1&0&1 &0\\
6 &1&1&0 &1\\
7 &1&1&1 &0\\
  \end{tabular}
\end{tabular}
\end{center}

\caption{Truth table of Equation \ref{eq:cansop}.}
\label{fig:tt}
\end{figure}

\begin{figure}[!hbt]
\begin{center}

\begin{tabular}{cc|c|c|}
a & b & \multicolumn{2}{c}{c} \\
& & \multicolumn{1}{c}{0} & \multicolumn{0}{c}{1} \\
\cline{3-4}
0 & 0  & \cellcolor{red!40}1 & 0 \\
\cline{3-4}
0 & 1  & \cellcolor{red!40!blue!40}1 & \cellcolor{blue!40}1 \\
\cline{3-4}
1 & 1  & \cellcolor{red!40}1 & 0 \\
\cline{3-4}
1 & 0  & \cellcolor{red!40}1 & 0 \\
\cline{3-4}
\end{tabular}

%\karnaughmap{3}{f(a,b,c):}{abc}{10111010}{}
% This produces a nice looking map but I dont understand the format.

\end{center}
\caption{Karnaugh Map of Equation \ref{eq:cansop}.
The largest 4 element grouping is shown in red and the smaller
two element grouping is shown in blue.}
\label{fig:karnmap}
\end{figure}

\begin{figure}[!hbtp]
\center
\includegraphics[scale=0.5]{circuit-01}
\caption{Circuit definition of equation \ref{eq:simp}.}
\label{fig:circuit}
\end{figure}


%The design of these logic functions implemented with gates is shown
%in Figure \ref{fig:minsoplog} and \ref{fig:maxposlog}.
%Interestingly, the only change made between the two diagrams is
%that of swapping the AND gates with OR gates and vice versa.

\clearpage

\section{Observations}

%The output of each logic function implemented in hardware agreed
%with the truth table.
%To switch between the minterm POS to maxterm SOP it was only
%necessary to swap the 7432 OR gate with the 7408 AND gate.
%This is possible because the layouts are identical for each chip.

\clearpage
\section{Conclusion}

%This lab was a success in showing the equivalence of minterm SOP
%form and maxterm POS form by implementing the function in hardware.

% flush all the figures
%\clearpage

%\pagebreak
%\renewcommand*{\refname}{\vspace{-8mm}}
%\section{References}
%\bibliographystyle{plain}
%\bibliographystyle{mslapa}
%\bibliographystyle{ieeetr}
%\bibliography{../references}

% Appendix (if needed)

\end{document}

% vim:foldmethod=marker
