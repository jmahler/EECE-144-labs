
% # COPYRIGHT:
%
%   Copyright (C)  2011 Jeremiah Mahler <jmmahler@gmail.com>.
%   Permission is granted to copy, distribute and/or modify this document
%   under the terms of the GNU Free Documentation License, Version 1.3
%   or any later version published by the Free Software Foundation;
%   with no Invariant Sections, no Front-Cover Texts, and no Back-Cover Texts.
%   A copy of the license is included in the file "fdl-1.3.txt".
%

\documentclass[12pt]{article}
%\documentclass[10pt]{article}

%\usepackage{mslapa}
\usepackage{hyperref}
\usepackage{amsmath}
\usepackage{graphicx}
\usepackage{ulem}
%\usepackage{vmargin}
\usepackage{tabularx}
\usepackage{sectsty}
\usepackage{pbox}
\usepackage{bigstrut}
\usepackage{enumerate}
\usepackage{parskip}  % add spaces between paragraphs
\input kvmacros  % Karnaugh Maps and Veitch charts
\usepackage[table]{xcolor}

%\usepackage{cleveref}

%\setpapersize{USletter}
\sectionfont{\normalsize}
\subsectionfont{\normalsize}

% configure \bigstrut size
% This configures spacing above and below rows in a tabularx.
%\renewcommand{\bigstrutjot}{6pt}
\renewcommand{\bigstrutjot}{2.0\jot}

\setlength{\parindent}{0in}

\raggedright

\begin{document}

% {{{ Cover Page

\centerline{\bf EECE 144}
\centerline{\bf Fall 2011}
\centerline{\bf}
\centerline{\bf Lab Report \#5}
\centerline{\bf Section 4}
\centerline{\bf 10/5/2011}

% signature area
\begin{center}
\begin{tabularx}{\textwidth}[b]{X l l}
Submitted by: Jeremiah Mahler & & \\
Signature & Printed Name & Date \\
\hline
\multicolumn{1}{|X|}{} & \multicolumn{1}{|l|}{\bigstrut \bf Jeremiah Mahler} & \multicolumn{1}{|l|}{\bf Oct 5, 2011} \\
\hline
\multicolumn{1}{|X|}{} & \multicolumn{1}{|l|}{\bigstrut \bf Marvanee Johnson} & \multicolumn{1}{|l|}{\bf Oct 5, 2011} \\
\hline
\end{tabularx}
\end{center}
% }}}

\section{Description/Objectives}

The objective of this lab is to derive a simplified form
of a canonical SOP equation using a Karnaugh Map and
to implement it in hardware.

\section{Procedure}
\label{sec:procedure}

The canonical SOP form of the equation used in this lab is
given in Equation \ref{eq:cansop}.

\begin{align}
f(a, b, c) &= \sum m(0, 2, 3, 4, 6) \label{eq:cansop} \\
	    &= m_0 + m_2 + m_3 + m_4 + m_6 \notag \\
		&= a' b' c' + a' b c' + a' b c + a b' c' + a b c' \notag
\end{align}

The truth table produced from Equation \ref{eq:cansop}
is shown in Figure \ref{fig:tt}.
And the Karnaugh Map produced from the truth table
is shown in Figure \ref{fig:karnmap}.
Using the groupings produced by the Karnaugh Map in Figure \ref{fig:karnmap}
results in the simplified form in Equation \ref{eq:simp}.
The implementation of the logic function using gates is
shown in Figure \ref{fig:circuit}.

\begin{align}
f(a, b, c) &= c' + a'b \label{eq:simp}
\end{align}

\begin{figure}[!hbt]
\begin{center}

\begin{tabular}{lr}
  \begin{tabular}[t]{r|ccc|c}
Index&$a$&$b$&$c$&$f$\\
\hline
0 &0&0&0 &1\\
1 &0&0&1 &0\\
2 &0&1&0 &1\\
3 &0&1&1 &1\\
4 &1&0&0 &1\\
5 &1&0&1 &0\\
6 &1&1&0 &1\\
7 &1&1&1 &0\\
  \end{tabular}
\end{tabular}
\end{center}

\caption{Truth table of Equation \ref{eq:cansop}.}
\label{fig:tt}
\end{figure}

\begin{figure}[!hbt]
\begin{center}

\begin{tabular}{c c c}

%\begin{tabular}{cc|c|c|}
%     & \multicolumn{1}{c}{$c$} & \multicolumn{2}{c}{} \\
%$ab$ &  \multicolumn{1}{c}{} & \multicolumn{1}{c}{0} & \multicolumn{1}{c}{1} \\
%	\cline{3-4}
%	& 00 & \cellcolor{red!40}1 & 0 \\
%	\cline{3-4}
%	& 01 & \cellcolor{red!40!blue!40}1 & \cellcolor{blue!40}1 \\
%	\cline{3-4}
%	& 11 & \cellcolor{red!40}1 & 0 \\
%	\cline{3-4}
%	& 10 & \cellcolor{red!40}1 & 0 \\
%	\cline{3-4}
%\end{tabular}

\begin{tabular}[b]{cc|c|c|c|c|}
     & \multicolumn{1}{c}{$ac$} & \multicolumn{4}{c}{} \\
$b$ &  \multicolumn{1}{c}{} & \multicolumn{1}{c}{00} & \multicolumn{1}{c}{01} & \multicolumn{1}{c}{11} & \multicolumn{1}{c}{10} \\
	\cline{3-6}
	& 0 & \cellcolor{red!40}1 & 0 & 0 & \cellcolor{red!40}1 \\
	\cline{3-6}
	& 1 & \cellcolor{red!40!blue!40}1 & \cellcolor{blue!40}1 & 0 & \cellcolor{red!40}1 \\
	\cline{3-6}
\end{tabular}

& &

\karnaughmap{3}{$f(a,b,c)$:}{abc}{10111010}{}

\end{tabular} % outer tabular

\end{center}
\caption{
Two equivalent Karnaugh Map representations of Equation \ref{eq:cansop}.
The first shows the explicit values of $a$, $b$, and $c$ along with
red and blue coloring to denote the groupings.
The second is a Veitch \cite{Veitch:1952:CMS:609784.609801} diagram.}
\label{fig:karnmap}
\end{figure}


\begin{figure}[!hbtp]
\center
\includegraphics[scale=0.5]{circuit-01}
\caption{Circuit definition of equation \ref{eq:simp}.}
\label{fig:circuit}
\end{figure}

\clearpage

\section{Observations}

The output of each logic function implemented in hardware
(Figure \ref{fig:circuit}) agreed with the expected values
of the truth table (Figure \ref{fig:tt}).

\section{Conclusion}

This lab was a success in showing that simplification of a
logic function using a Karnaugh map produces produces an
equivalent function which can be implemented in hardware.

% flush all the figures
%\clearpage

%\pagebreak
\renewcommand*{\refname}{\vspace{-8mm}}
\section{References}
%\bibliographystyle{plain}
%\bibliographystyle{mslapa}
\bibliographystyle{ieeetr}
\bibliography{../references}

% Appendix (if needed)

\end{document}

% vim:foldmethod=marker
