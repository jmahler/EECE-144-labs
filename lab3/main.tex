
%
% # INTRODUCTION:
%
% This write up is for Lab3: Logic Implementation Using IC's
%
% Specifically this was used for the class Logic Design Fundamentals (EECE 144)
% taught by Kurtis Kredo [http://www.ecst.csuchico.edu/~kkredo/]
% during the Fall 2011 semester at CSU Chico [www.csuchico.edu].
% 
% ## LaTeX
%
% This file is written for LaTeX [http://www.latex-project.org/]
% which is used to process this file in to a completely formatted
% document.
%
% If you are unfamiliar with LaTeX it can seem daunting at first
% (as with anything new) but there are many benefits.
% Imagine writing a document in Word except without
% having to worry about the tedious things such as line breaks,
% indentation, table of contents, appendices, font styles,
% heading sizes, citations/references, page numbers.
% LaTeX lets you focus on the content without
% worrying about the tedious details.  It is also excellent for
% producing mathematical formulas.
% 
% If you are collaborating with someone else you can simply edit
% the sections and paragraphs in this file as needed.
%
% To process this file use a command such as 'rubber'.
%
%   bash$ rubber skel.tex
%   (output to skel.dvi)
%   bash% rubber --pdf skel.tex
%   (output to skel.pdf)
%
% # AUTHORS (of this template):
%
%   Jeremiah Mahler <jmmahler@gmail.com>
%   https://www.google.com/profiles/jmmahler#about 
%
% # COPYRIGHT:
%
%   Copyright (C)  2011 Jeremiah Mahler <jmmahler@gmail.com>.
%   Permission is granted to copy, distribute and/or modify this document
%   under the terms of the GNU Free Documentation License, Version 1.3
%   or any later version published by the Free Software Foundation;
%   with no Invariant Sections, no Front-Cover Texts, and no Back-Cover Texts.
%   A copy of the license is included in the file "fdl-1.3.txt".
%

\documentclass[12pt]{article}
%\documentclass[10pt]{article}

%\usepackage{mslapa}
\usepackage{hyperref}
\usepackage{amsmath}
\usepackage{graphicx}
\usepackage{ulem}
%\usepackage{vmargin}
\usepackage{tabularx}
\usepackage{sectsty}
\usepackage{pbox}
\usepackage{bigstrut}
\usepackage{enumerate}

%\usepackage{cleveref}

%\setpapersize{USletter}
\sectionfont{\normalsize}
\subsectionfont{\normalsize}

% configure \bigstrut size
% This configures spacing above and below rows in a tabularx.
%\renewcommand{\bigstrutjot}{6pt}
\renewcommand{\bigstrutjot}{2.0\jot}

\setlength{\parindent}{0in}

\raggedright

\begin{document}

% {{{ Cover Page

\centerline{\bf EECE 144}
\centerline{\bf Fall 2011}
\centerline{\bf}
\centerline{\bf Lab Report \#3}
\centerline{\bf Section 4}
\centerline{\bf 9/21/2011}

% signature area
\begin{center}
\begin{tabularx}{\textwidth}[b]{X l l}
Submitted by: & & \\
Signature & Printed Name & Date \\
\hline
\multicolumn{1}{|X|}{} & \multicolumn{1}{|l|}{\bigstrut \bf Jeremiah Mahler} & \multicolumn{1}{|l|}{\bf Sep 21, 2011} \\
\hline
\multicolumn{1}{|X|}{} & \multicolumn{1}{|l|}{\bigstrut \bf Marvanee Johnson} & \multicolumn{1}{|l|}{\bf Sep 21, 2011} \\
\hline
\end{tabularx}
\end{center}
% }}}

\section{Description/Objectives}

\section{Procedure}

(Equations \ref{eq:fn1})
\nocite{LOGISIM}

\begin{align}
& a c + a' b + a b' c' \label{eq:fn1}
\end{align}

\begin{figure}[!hbt]

\center

\begin{tabular}[t]{| l | l | l | l |}
\hline
\multicolumn{4}{| c |}{$a c + a' b + a b' c'$} \\
\hline
$a$ & $b$ & $c$ & z (out) \\
\hline
0 & 0 & 0 & ? \\
\hline
0 & 0 & 1 & ? \\
\hline
0 & 1 & 0 & ? \\
\hline
0 & 1 & 1 & ? \\
\hline
1 & 0 & 0 & ? \\
\hline
1 & 0 & 1 & ? \\
\hline
1 & 1 & 0 & ? \\
\hline
1 & 1 & 1 & ? \\
\hline
\end{tabular}

\caption{Truth table of outputs for the function $a c + a' b + a b' c'$.}
\label{fig:out1}
\end{figure}

\section{Observations}

%\clearpage

\section{Conclusion}

% flush all the figures
%\clearpage

%\pagebreak
\renewcommand*{\refname}{\vspace{-8mm}}
\section{References}
%\bibliographystyle{plain}
%\bibliographystyle{mslapa}
\bibliographystyle{ieeetr}
\bibliography{../references}

% Appendix (if needed)

\end{document}

% vim:foldmethod=marker
